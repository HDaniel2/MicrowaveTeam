\documentclass[12pt,a4paper]{article}
\usepackage[utf8]{inputenc}
\usepackage[german]{babel}
\usepackage[T1]{fontenc}
\usepackage{amsmath}
\usepackage{amsfonts}
\usepackage{amssymb}
\usepackage{lscape}
\usepackage{graphicx}
\usepackage{multirow}
\usepackage[left=2cm,right=2cm,top=2cm,bottom=2cm]{geometry}
\begin{document}

\tableofcontents

\newpage

\section{Typischer Ablauf: Optische Litographie}
\begin{center}
\textbf{Vorbereitung} (Siehe 2.)\\
$\downarrow$\\
Probe unter Mikroskop begutachten\\
$\downarrow$\\
\textbf{Probe reinigen} (Siehe 3.)\\
\bigskip
\textit{Währendessen:}\\
\textbf{Maske einbauen/wechseln} (Siehe 4.)\\
$\downarrow$\\
\textbf{Lack auftragen} (Siehe 5.)\\
$\downarrow$\\
\textbf{Probe belichten} (Siehe 6.)\\
\textit{Falls nötig auch eine Randentlackung machen}\\
$\downarrow$\\
\textbf{Photolack entwickeln} (Siehe 7.)\\
$\downarrow$\\
Probe unter Mikroskop begutachten\\
$\downarrow$\\
\textbf{Höhenprofil aufnehmen} (Siehe 8.)\\
$\downarrow$\\
\textbf{Fertig}
\end{center}



\section{Vorbereitung}
\begin{itemize}
\item \textbf{Handschuhe} und \textbf{Mundschutz} anziehen.
\item \textbf{Stickstoff} und \textbf{Druckluft} im Nebenraum (1.120, Service 1) aufdrehen
\item \textbf{Belichtungsmaschine} einschalten (Braucht ca. 20min bis die Lampe aufgeheizt ist)

\begin{description}
\item 1) \textbf{On} drücken $\rightarrow$ Fertig bei \textbf{ready}
\item 2) \textbf{CP} drücken (constant power)
\item 3) \textbf{Start} drücken $\rightarrow$ \textit{Dauert nun ca. 20 min}\\
\textit{Braucht später nach dem Ausschalten erneut ca. 25 min bis abgekühlt ist}
\end{description}

\item \textbf{Vakuumpumpe} einschalten
\item \textbf{Pinzette} mit Aceton und anschließend mit Iso-Propanol und jeweils einem Reinraumtuch säubern.
\item \textbf{Heizplatte} einschalten. Anschließend die benötigte Zieltemperatur durch Drücken der \textbf{Set}-Taste (gedrückt halten) und den \textbf{Pfeiltasten} einstellen.
\end{itemize}

\newpage

\section{Probe reinigen}
\begin{center}
\begin{tabular}{| l | p{10cm} |} \hline
\textbf{5 Kleine Gläschen} & 2 für Aceton und 3 für Iso-Propanol\\
& (Gläschen beschriften bzw. irgendwie unterscheiden!)\\ \hline
\textbf{1 Großes Glas} & Für das Aceton und Iso-Propanol Rest-Gemisch (Waste)\\ \hline
\textbf{Pinzette} & \\
\textbf{Reinraumtücher} & \\
\textbf{Schutzbrille} & \\ \hline
\end{tabular}
\end{center}

\bigskip

\begin{center}
\textbf{Reinigungsprozess:}
\end{center}
\begin{tabular}{l p{14cm}}
\underline{Bemerkung:} & Eventuell vor und nach dem Reinigen \textbf{Bilder vom Chip am Mikroskop abspeichern}, um später nachverfolgen zu können, woher möglicher Dreck/Fehlstellen kommen könnten.
\end{tabular}


\begin{description}
\item 1) \textbf{5 kleine Gläschen} nehmen und mit 2 mit Aceton bzw. 3 mit Iso-Propanol und einem \textbf{Reinraumtuch} reinigen.

\item 2) Probe mit \textbf{Pinzette} nehmen, Oberfläche \textbf{dauerhaft mit Aceton benetzen} und währenddessen in das Aceton-Gläschen legen. Dieses anschließend ein wenig mit Aceton \textbf{auffüllen}, bis die Probe bedeckt ist.\\ 
\underline{\textbf{Wichtig:}} Niemals das Aceton am Chip trocknen lassen ($\rightarrow$ Sonst \textbf{Schlierenbildung})
\begin{center}
\textit{Schritt 2 einmal wiederholen}
\end{center}

\item 3) Probe mit \textbf{Pinzette} nehmen, Oberfläche \textbf{dauerhaft mit Iso-Propanol benetzen} und währenddessen in das Iso-Gläschen legen. Dieses anschließend ein wenig mit Iso \textbf{auffüllen}, bis die Probe bedeckt ist.\\ 
\underline{\textbf{Wichtig:}} Auch hier das Isopropanol nicht auf der Probe trocken werden lassen.
\begin{center}
\textit{Diesen 3 zweimal wiederholen}
\end{center}

\item 5) Probe aus Gläschen nehmen, dabei etwas Iso-Propanol \textbf {abspühlen}. Dann zügig und vorsichtig auf Reinraumtuch legen, gut festhalten und mit \textbf{Stickstoff trocken blasen}.\\

\item 6) Probe auf \textbf{Heizplatte} legen und für \textbf{ca. 10s} (Bei Silizium) bzw. \textbf{ca. 10-20s} (sonst) trocknen.\\

\item 7) Abschließend die Probe unter dem \textbf{Mikroskop begutachten}. Falls Unreinheiten erkennbar $\rightarrow$ Eventuell zunächst Isopropanol abspühlen und erneut trocken blasen.\\

Falls Schmutz immernoch vorhanden $\rightarrow$ \textbf{Reinigung wiederholen...} (Zuvor benutzte Acetongläschen erneut reinigen!)
\end{description}

\newpage

\section{Maske einbauen bzw. wechseln}
\begin{center}
\textbf{\underline{Vorbereitung}}
\end{center}
\begin{description}
\item i) Die \textbf{neu einzubauende Maske} unter dem Mikroskop begutachten. Falls nötig mit Aceton, Iso-Propnaol und evtl. Reinraumtüchern reinigen.
\item ii) Beide \textbf{seitliche große Rädchen} am Mask-Aligner auf 10 stellen (Zentriert den Probentisch)
\item iii) \textbf{Vordere kleine herausstehende Platte} durch vorderes, kleines Rädchen (auf der rechten Seite) mittig positionieren (Ermöglicht Rotation um z-Achse).
\item iv) \textbf{MaskAligern starten:\footnotemark[1]}\\ 
(Grüner Kippschalter auf \textbf{On} kippen $\rightarrow$ \textbf{Load} drücken $\rightarrow$ \textit{Miksoskop fährt herunter)}
\end{description}

\footnotetext[1]{Der MaskAligner lässt sich nicht starten, wenn die Lampe nicht aufgeheizt ist. Daher muss diese zuvor (Am besten zu Beginn der Litho) eingeschaltet werden.}
\bigskip

\begin{center}
\textbf{\underline{Einbauen bzw. wechseln}}
\end{center}

\begin{description}

\item 1) \textbf{CHANGE MASK} drücken $\rightarrow$ \textit{Mikroskop fährt hoch}

\begin{description}

\item \underline{Beim Wechseln...}
\item 2) Maskenhalter herausziehen und umgedreht links auf der Ablage ablegen (Dabei eine Hand unter die Maske halten, falls sie runterfallen sollte)\newline
\textbf{Enter} drücken (entfernt Vakuum) $\rightarrow$ Alte Maske entfernen

\item \underline{Beim Einbauen...}
\item 2) Kleinen (bzw. großen) Maskenhalter aus Schrank nehmen und links umgedreht auf die Ablage legen. Vakuumschlauch anschließen.

\end{description}

\item 3) \textbf{Neue Maske} mit der dunklen Seite (Chrom-Seite) nach oben so \textbf{positionieren}, dass gewünschte Struktur möglichst mittig ist und die Maske von genügend Löchern für das spätere Vakuum angesogen werden kann.

\item 4) \textbf{Enter} drücken $\rightarrow$ \textit{Aktiviert (Deaktiviert) das Vakuum}\\
\textbf{Wichtig:} Testen, ob Maske wirklich korrekt angezogen wird!

\item 5) \textbf{Maskenhalter} vorsichtig wieder einsetzen und währendessen mit einer Hand die angesogene Maske vor einem möglichen Fall schützen.

\item 7) \textbf{CHANGE MASK} drücken $\rightarrow$ Fertig
\end{description}

\newpage

\section{Lack auftragen}

\begin{tabular}{| p{5cm} | p{10cm} |} \hline
\textbf{Pipette und Pipettenaufsätzen} & Befinden sich im Glas-Regal rechts hinten im Raum\\ \hline
\textbf{Benötigter Lack} & Befindet sich auf im Glas-Regal.\newline
\textbf{Achtung:} Lack vorsichtig und langsam rübertragen!\\ \hline
\textbf{Großes Glas} & Zum Zwischenlagern der Pipette\\ \hline
\textbf{Metallplatte} & Zum Abkühlen des Chips und damit sich der Lack anpassen kann\\ \hline
\textbf{Spincoater} und \textbf{Heizplatte} & Heizplatte sollte bereits nach Betreten des Litho-Raums auf die gewünschte Temperatur gestellt werden.\\ \hline

\end{tabular}

\bigskip

\begin{description}
\item 1) \textbf{Spincoater zusammenbauen}

\item 2) \textbf{Schleuder einschalten} (grüner Kippschalter)\newline
$\rightarrow$ Warten, bis alle Tests durchlaufen sind und die Fehlermeldung \textbf{Empty Battery} erscheint\newline
$\rightarrow$ Zum Quittieren \textbf{Enter} drücken

\item 3) \textbf{Rezept} einprogrammieren:\newline
\textbf{Edit} $\rightarrow$ (Edit Menu) $\rightarrow$ \textbf{Edit} $\rightarrow$ (A blinkt) $\rightarrow$ mit \textbf{Shift} rechts auf \textbf{Edit} (Um das Rezept zu ändern) $\rightarrow$ \textbf{Enter} $\rightarrow$ \textbf{EDIT} $\rightarrow$ \textbf{Enter} $\rightarrow$ Gewünschte Parameter eingeben (Jeweils mit \textbf{Enter} bestätigten)\\
\begin{center}
\begin{tabular}{|l | c | l|}\hline
\textbf{Parameter} & \textbf{Beispiel} & \textbf{Beschreibung}\\ \hline
rpm & 6000 & rotations per minute\\
Ramp & 10 & Zeit, um auf gewünschte rpm zu kommen\\
Out & & Keine Angabe\\
Time & 40 & Gesamte Schleuderzeit\\
E & n & Keine Angabe\\ \hline
\end{tabular}
\end{center}

\textbf{Zurück-Symbol} drücken $\rightarrow$ (Main MENU) $\rightarrow$ \textbf{Run} $\rightarrow$ \textbf{Run} $\rightarrow$ \textit{Rezept A auswählen} $\rightarrow$ Mit \textbf{Enter} bestätigten

\item 4) \textbf{Lackschleuder Vakuum} drücken $\rightarrow$ \textit{Vakuum-Pumpe wird eingeschaltet}

\item 5) Rezept zunächst mit Dummy-Chip überprüfen
\begin{description}
\item i) Gummiring mittig platzieren und Dummy-Chip darauf legen 
\item ii) \textbf{Vakuum} drücken und überprüfen, dass Chip angesogen wird
\item iii) Mit \textbf{Start} oder \textbf{F1} Rezept starten und dabei Parameter überprüfen

\end{description}

\item 6) Aufsatz auf Pipette stecken, die Spitze mit Stickstoff reinigen, die zu entnehmende Menge durch Drehen an der Pipette einstellen und den Lack vorsichtig in die Pipette füllen. $\rightarrow$ Pipette auf größerem Glas zwischenlagern und Lackglas verschließen

\item 7) Etwas vom Lack aus Pipette entfernen, um Luftblasen zu vermmeiden, anschließend zügig den Lack auf die Probe auftragen und Rezept durch \textbf{Start} oder \textbf{F1} starten.

\item 8) Anschließend mit bloßem Auge und/oder am Mikroskop (Mit \textbf{Rotfilder}!) Lack auf glatte Oberfläche überprüfen\newline
\textbf{Hinweis:} Sollte der Lack Unregelmäßigkeiten (z.B. Schlieren) aufweisen, muss der Lack entfernt werden (Mit Aceton oder speziellen Lackentfernern), die Probe nochmal gereinigt werden und der Lack erneut aufgetragen werden.

\item 9) Sieht die Lackoberfläche gut aus, die Probe \textbf{einige Minuten} (Abhängig vom Lack $\rightarrow$ evtl. Liste vor Ort berücksichtigen) auf die Heizplatte legen und anschließend erneut \textbf{ca. 2-5min} auf der Kühlplatte auskühlen lassen.

\end{description}

\newpage

\section{Chip und Maske angleichen und belichten}
\begin{center}
\begin{tabular}{| l | c | l |}
\multicolumn{3}{c}{Unter \textbf{EDIT PARAMETER} veränderbare Einstellungen}\\ \hline
& Beispiel & \\ \hline
\textbf{Process:}  & Lithography & Prozess\\ \hline
\textbf{Exp. Time[s]:} & 4,5 & Belichtungszeit\\ \hline
\textbf{Al. Gap[$\mu m$]:} & 200 & Abstand \textit{Probe - Maske} 
während des Anpassens\\ \hline
\textbf{Expose Type:} & Hard & Probe wird an Maske gepresst\\ \hline
\textbf{HC Wait T.[s]:} & 2 & Keine Angabe\\ \hline
\textbf{WEC Type:} & Cont & Wedge Error Compensation\\ \hline
\textbf{N2 Purge:} & No & Yes : Stickstoff, No: Kein Stickstoff\\ \hline
\textbf{WEC Offset:} & Off & Keine Angabe\\ \hline
\end{tabular}
\end{center}

\begin{description}
\item 1) \textbf{Merken}, wo sich das benötigte Bild auf der Maske befindet.

\item 2) \textbf{Load} drücken $\rightarrow$ Schublade bis zum Anschlag herausziehen

\item 3) Probe nun \textbf{grob positionieren}, sodass Probe Löcher überdeckt und später angesogen werden kann

\item 4) \textbf{Enter} drücken $\rightarrow$ \textit{Probe wird angesogen (Dies auch überprüfen!)}

\item 5) Schubladen wieder einfahren und grob von oben schauen, ob Probe gut genug positioniert ist\newline

\underline{Sollte Probe und Bild zu weit entfernt liegen:}\\ 
\textbf{Probe neu positionieren:} \textbf{Unload} drücken $\rightarrow$ Schublade rausziehen $\rightarrow$ \textbf{Enter} drücken (Vakuum wird aufgehoben) $\rightarrow$ Position korrigieren $\rightarrow$ \textbf{Enter} drücken (Vakuum wird eingeschaltet) $\rightarrow$ \textbf{Load} drücken $\rightarrow$ Schublade einfahren\\
\textbf{Eventuell Maske neu positionieren:} Siehe 4.

\item 6) (Wenn Probe gut positioniert ist) $\rightarrow$ \textbf{Enter} drücken \newline 
$\rightarrow$ \textit{Probe wird an die Maske gepresst und dann den in \textbf{Al. Gap [$\mu m$]} gewählten Abstand von ihr entfernt.}

\textbf{Hinweis:} Sollte das Licht bei \textbf{BSA MICROSCOPE} leuchten,  sollte dieses nun ausgeschaltet werden.

\item 7) In das Mikroskop schauen und mit Pfeiltasten die Probe und das gewünschte Bild suchen (Geht schneller, wenn die Taste \textbf{FAST} gedrückt wurde). DIe Probe mit den seitlichen großen 2 Rädchen in x- und y-Richtung verschieben und mit dem kleineren Rädchen vorne rechts so um die z-Achse rotieren, bis Maske und Probe wie gewünscht übereinstimmen.\newline
\textbf{Hinweis:} Auf bereits bestehende Identifier achten, sollte einer vorhanden sein (Evtl. muss Probe neu positioniert werden $\rightarrow$ Siehe Schritt 5) )

\item 8) Mit \textbf{ALIGNN CONT/EXP} wird die Probe (Wie während des Belichtens auch) gegen die Maske gepresst. Nun kann überprüft werden, dass die Probe nicht beim Andrücken (Aufgrund unterschiedlich hoher Lackberge am Rand) verrutscht. ($\rightarrow$ Erneut drücken, damit Probe wieder entfernt wird)

\item 9) Liegt Probe wie gewünscht $\rightarrow$ \textbf{Exposure} drücken $\rightarrow$ \textit{Startet die Belichtung}
\begin{center}
\textbf{Nicht in die austretende UV-Strahlung blicken!}
\end{center}

\item 10) \textbf{Unload} drücken $\rightarrow$ Schublade herausziehen $\rightarrow$ \textbf{Enter} drücken (Entfernt Sog) $\rightarrow$ Probe entfernen $\rightarrow$ Maske (Falls nicht mehr benötigt) wie in \textbf{4. Maske wechseln} beschrieben entfernen.

\end{description}


\section{Entwickeln des Photolacks}

\begin{tabular}{| p{5cm} | p{10cm} |} \hline
\textbf{Entwickler} & Ist abhängig vom verwendeten Lack\newline
$\rightarrow$ \textit{Siehe Liste vor Ort, welche Entwickler beim benutzten Lack verwendet wurden}\\ \hline
\textbf{2 Kleine Gläschen} & 1 Gläschen für den Entwickler und 1 Gläschen für hochreines Wasser\\ \hline
\textbf{Stoppuhr} & \\
\textbf{Schutzbrille!} & \\ \hline
\end{tabular}


\begin{description}


\item 0) (Eventuell Temperatur und Luftfeuchtigkeit im Raum notieren)

\item 1) Beide Gläser mit Reinstwasser reinigen und eines mit selbigem auffüllen

\item 3) Anderes Gläschen mit Entwickler auffüllen (komplett, damit genügend Entwickler vorhanden ist) und beide Gläser beschriften bzw. markieren.

\item 4) Probe im Entwickler (Eventuell hochkant) eine gewisse Zeit (Vergleiche Liste vor Ort) gut \textbf{schwenken}, sodass die Probe permanent mit dem Entwickler in Kontakt kommt (Entwickler wird währendessen teilweise verbraucht)

\item 5) Anschließend Probe im Wasserglass kurz ausspülen und anschließend mit Stickstoff trocken blasen (Und eventuell 5-10 s auf Heizplatte trocknen lassen)

\item 6) Unter dem Mikroskop überprüfen, ob Struktur korrekt übertragen wurde und ob sich Fehler/Dreck gebildet haben (Falls grobe Fehler: Lack mit Aceton entfernen, ca. 15s auf Heizplatte trocknen lassen und Lack erneut auftragen)

\end{description}

\newpage

\section{Höhenprofil (DektakXT)}

\begin{description}
\item 0) Falls Computer und Gerät ausgeschaltet sind:
\begin{description}
\item i) Gerät durch \textbf{weißen Schalter} (Zwischen Gerät und Computerbildschirm) \textbf{einschalten}
\item ii) Computer starten und \textbf{Benutzer: Profilometer-User} auswählen (Das Passwort ist exakt der Name des Geräts)
\item iii) Programm \textbf{Vision 64 starten} ($\rightarrow$ \textit{Gerät initialisiert sich, es muss immer mal OK gedrückt werden})
\end{description}

\item 1) Probe auf Probenteller legen, Lucke schließen und \textbf{Load Sample} drücken

\item 2) \textbf{Tower down} drücken $\rightarrow$ \textit{Spitze fährt auf Probe herunter}

\underline{Probe abfahren:} Im Programm auf der rechten Seite kann man die Probe in x-y-Richtung (\textbf{XY}) abfahren (Die geht manuell mit verschieben des \textbf{roten Punktes auf dem Fadenkreuz} oder durch angeben einer Speziellen Position als \textbf{X- und Y-Koordinate})\\

\underline{Probe drehen:} Unterhalb von \textbf{XY} kann die Probe um den Winkel (\textbf{Theta}) manuell (Durch drücken der gelben Dreiecke) gedreht werden. Es kann aber auch ein bestimmter Winkel eingegeben werden (bspw. 90 Grad oder 180 Grad), um des sich die Probe dann dreht.

\item 4) \textbf{Measurement Setup} drücken $\rightarrow$ Anschließend links den \textbf{Zauberstab} drücken.

\item 5) Zur \textbf{Startposition} fahren und diese mit \textbf{Next} bestätigen
\item 6) Zur \textbf{Endposition} fahren und diese mit \textbf{Next} bestätigen

\item 7) Die gewünschte \textbf{Dauer} unter \textbf{Duration:} \textbf{einstellen}
\item 8) \textbf{Die Messung} nun durch drücken des grünen Startknopfes (\textit{Measurement}) \textbf{starten} und \textbf{Kontakt mit dem Tisch vermeiden}, um Messung nicht zu behindern.

\begin{center}
\underline{\textbf{Gemessene Daten bearbeiten}}
\end{center}

\begin{description}
\item 1) Sollte der Verlauf schräg sein:
\begin{description}
\item i) Rechtsklick $\rightarrow$ Bereich R hinzufügen
\item ii) Rechtsklick $\rightarrow$ Bereich M hinzufügen
\item iii) Mit beiden Bereichen einen Teil der Messdaten auswählen, der als eben (horizontal) angenommen wird
\item iv) Durch drücken des Doppelpfeils (rechts oben) werden Daten angepasst
\end{description}

\item 2) Befindet sich der M und R Bereich auf unterschiedlichen Höhen, wird rechts unten der Höhenunterschied angegeben

\begin{center}
\underline{\textbf{Daten speichern}}
\end{center}

\begin{description}
\item 1) Daten als .csv Datei abspeichern (Rechtsklick $\rightarrow$ \textit{Export data})

\item 2) Messdatei speichern, um Daten im Programm später wieder aufgrunfen zu können (Links oben auf \textbf{Save})

\item 3) Bild des Höhenprofils abspeichern (Rechts oben unter \textbf{Save as bitmap} oder so ähnlich)
\end{description}

\end{description}

\end{description}


\begin{landscape}

\section{Übersichtstabelle}

\begin{tabular}{| l | l | l || l | l | l | l | l | l | l | l | l || p{7cm} |}\hline
\multirow{2}{*}{\textbf{Datum}} & 
\multirow{2}{*}{\textbf{Probe}} &  
\multirow{2}{*}{\textbf{Maske}} & 
\multirow{2}{*}{\textbf{Lack}} & 
\textbf{Ent-} & 
\multirow{2}{*}{$T_{Heiz}$} & 
\multicolumn{3}{| c |}{\textbf{Dauer [s]}} & 
\multicolumn{3}{| c |}{\textbf{Rezept}} & 
\multirow{2}{*}{\textbf{Kommentar}}\\ \cline{7-12}
& & & & \textbf{wickler} & & Heizpl. & Belicht. & Entw. & rpm & ramp & time & \\ \hline \hline
&&&&&&&&&&&&\\
&&&&&&&&&&&&\\ \hline
&&&&&&&&&&&&\\
&&&&&&&&&&&&\\ \hline
&&&&&&&&&&&&\\
&&&&&&&&&&&&\\ \hline
&&&&&&&&&&&&\\
&&&&&&&&&&&&\\ \hline
&&&&&&&&&&&&\\
&&&&&&&&&&&&\\ \hline
&&&&&&&&&&&&\\
&&&&&&&&&&&&\\ \hline
&&&&&&&&&&&&\\
&&&&&&&&&&&&\\ \hline
&&&&&&&&&&&&\\
&&&&&&&&&&&&\\ \hline
&&&&&&&&&&&&\\
&&&&&&&&&&&&\\ \hline
&&&&&&&&&&&&\\
&&&&&&&&&&&&\\ \hline
&&&&&&&&&&&&\\
&&&&&&&&&&&&\\ \hline
&&&&&&&&&&&&\\
&&&&&&&&&&&&\\ \hline
&&&&&&&&&&&&\\
&&&&&&&&&&&&\\ \hline
&&&&&&&&&&&&\\
&&&&&&&&&&&&\\ \hline
&&&&&&&&&&&&\\
&&&&&&&&&&&&\\ \hline
&&&&&&&&&&&&\\
&&&&&&&&&&&&\\ \hline


\end{tabular}

\end{landscape}


\end{document}